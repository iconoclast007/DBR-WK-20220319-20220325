\section{Psalm 82 Comments}

\subsection{Numeric Nuggets}
There are 13 words in verses 1 and 7, along with 13 unique words in verse 2. The 13$^{th}$ word in the psalm is ``gods.''  The 26$^{th}$ (2 x 13) word in the psalm is ``wicked.'' The 13$^{th}$ word in verse 7 is ``princes.''

\subsection{Introduction}
The context of Psalm 82 (as seen in verses 1, 5-6, and 8) is the Second Advent of Jesus Christ and these mysterious creatures called ``gods''.  Does the psalm actually  acknowledge that there are other deities with and despite whom the Lord works and over whom He presides and rules? 

Hints to the context of the psalm include the use of the word ``Selah'' in verse 2, a call for judgment against the wicked, a description of the ``foundations of the earth'', a mention of ``gods'', and a mention of the inheritance of God in verse 8. In the middle of the conflict, the victims, are the poor and fatherless, and the afflicted and needy, the poor and needy in verses 3 and 4. 

The ``fatherless' are first seen in Exodus 22:22-24, as a group which has a unique concern of the Lord.\footnote{\textbf{Exodus 22:22-24} - Ye shall not afflict any widow, or fatherless child. [23] If thou afflict them in any wise, and they cry at all unto me, I will surely hear their cry; [24 ] And my wrath shall wax hot, and I will kill you with the sword; and your wives shall be widows, and your children fatherless.} They are last seen in scripture in the tribulation context in James 1:27.\footnote{\textbf{James 1:27} - Pure religion and undefiled before God and the Father is this, To visit the fatherless and widows in their affliction, and to keep himself unspotted from the world.} The ``fatherless' are found in, notably, 13 books in the AV. 

Psalm 82 presents an accusation against God. He is judging among the ``gods'' but judging unjustly. The persons of the wicked are afflicting people and are apparently being allowed to do it.  But the situation is temporary, and the godly order of things will be restored.

\subsection{Psalm 82:1}
The meaning of ``the wicked'' is established in Job 8:22 (among the 103 references in the AV) with a specific dwelling place. They are the children of the ``wicked one'' in Matthew 13:38.\footnote{\textbf{Matthew 13:38} - he field is the world; the good seed are the children of the kingdom; but the tares are the children of the wicked one;} They are ``children of pride'' in Job 41:34.\footnote{\textbf{Job 41:34} - He beholdeth all high things: he is a king over all the children of pride.} The word ``mighty'' in the verse makes an immediate connection back to the mighty men in Genesis 6:4.\footnote{\textbf{Genesis 6:4} - There were giants in the earth in those days; and also after that, when the sons of God came in unto the daughters of men, and they bare children to them, the same became mighty men which were of old, men of renown.} Another connection is to Nimrod in Genesis 10.\footnote{\textbf{Genesis 10:8-9} - And Cush begat Nimrod: he began to be a mighty one in the earth. [ 9] He was a mighty hunter before the LORD: wherefore it is said, Even as Nimrod the mighty hunter before the LORD.}

The short answer to the question posed earlier in ``yes.'' 

\subsection{Psalm 82:2}
The question ``how long'' is one of the famous questions contained in scripture, asked first to Pharaoh In Exodus 10:3, and last in the Great tribulation in Revelation 6:2.\footnote{\textbf{Revelation 6:2} - And they cried with a loud voice, saying, How long, O Lord, holy and true, dost thou not judge and avenge our blood on them that dwell on the earth?} ``Unjustly'' is used twice, here and in Isaiah 26:16, describing favour and mercy being shown to the wicked in an effort to unsuccessfully stimulate repentance.\footnote{\textbf{Isaiah 26:10} - Let favour be shewed to the wicked, yet will he not learn righteousness: in the land of uprightness will he deal unjustly, and will not behold the majesty of the LORD.}

\subsection{Psalm 82:5}
The phrase ``most High'' is first used in the story of Abraham and Melchisedek in Genesis 14.\footnote{\textbf{Genesis 14:18-22} - And Melchizedek king of Salem brought forth bread and wine: and he was the priest of the most high God. [19] And he blessed him, and said, Blessed be Abram of the most high God, possessor of heaven and earth: [20 ]And blessed be the most high God, which hath delivered thine enemies into thy hand. And he gave him tithes of all. [21] And the king of Sodom said unto Abram, Give me the persons, and take the goods to thyself. [22] And Abram said to the king of Sodom, I have lift up mine hand unto the LORD, the most high God, the possessor of heaven and earth, }


\subsection{Psalm 82:6, 7}
Who are these ``gods''? For starters, verse 7 distinguishes them from ``men'' and from ``princes.'' It may well be that this psalm may be a pre-Deluxe psalm, composed by Enoch. This is supported by the fact that the death referred to was the be Noah's flood. The ``gods'' die like mean because they have ``given up their first estate'' spoken of in Jude 6 and 7.\footnote{\textbf{Jude 6,7} - And the angels which kept not their first estate, but left their own habitation, he hath reserved in everlasting chains under darkness unto the judgment of the great day. [7] Even as Sodom and Gomorrha, and the cities about them in like manner, giving themselves over to fornication, and going after strange flesh, are set forth for an example, suffering the vengeance of eternal fire.} These are the ``sons of God'' in Genesis 6, evidently judges, and judging badly. For other evidence, see Genesis 3:5.\footnote{\textbf{Genesis 3:5} - For God doth know that in the day ye eat thereof, then your eyes shall be opened, and ye shall be as gods, knowing good and evil.} See Exodus 15:11.\footnote{\textbf{Exodus 15:11} - Who is like unto thee, O LORD, among the gods? who is like thee, glorious in holiness, fearful in praises, doing wonders?} See Psalm 86:8, 95:3,and 136:2.\footnote{\textbf{Psalm 86:8} - (Psalms 86:8) "Among the gods there is none like unto thee, O Lord; neither are there any works like unto thy works."}\footnote{\textbf{Psalm 95:3} - For the LORD is a great God, and a great King above all gods.}\footnote{\textbf{Psalm 136:2} - O give thanks unto the God of gods: for his mercy endureth for ever.}

The verses confront the standard interpretation that these ``gods'' were human judges, given the titles as God's representatives. The standard view is the same nonsense as the ``godly line of Seth'' in Genesis 4 and 5, a view regarded as absurd by John Calvin.

The ``gods'' are among the ``principalities and powers'' of Ephesians 6:12.\footnote{\textbf{Ephesians 6:12} - For we wrestle not against flesh and blood, but against principalities, against powers, against the rulers of the darkness of this world, against spiritual wickedness in high places.}

\subsection{Psalm 82:8}
The phrase ``Arise, O God'' or ``Arise, O LORD'' is found 9 times in scripture, and is anything but symbolic or metaphoric. In each instance, it refers to judgment and deliverance. Notably, the bookends for the references refer to thy resting place or rest, pointing to the Millennium:
\begin{compactenum}[1.]
	\item  \textbf{2 Chronicles 6:41} Now therefore arise, O LORD God, into thy resting place, thou, and the ark of thy strength: let thy priests, O LORD God, be clothed with salvation, and let thy saints rejoice in goodness.
	\item \textbf{Psalm 3:7} Arise, O LORD; save me, O my God: for thou hast smitten all mine enemies upon the cheek bone; thou hast broken the teeth of the ungodly.
	\item \textbf{Psalm 7:6} Arise, O LORD, in thine anger, lift up thyself because of the rage of mine enemies: and awake for me to the judgment that thou hast commanded
	\item \textbf{Psalm 9:19}  Arise, O LORD; let not man prevail: let the heathen be judged in thy sight.
	\item \textbf{Psalm 10:12} Arise, O LORD; O God, lift up thine hand: forget not the humble.
	\item \textbf{Psalm 17:13} Arise, O LORD, disappoint him, cast him down: deliver my soul from the wicked, which is thy sword: 
	\item \textbf{Psalm 74:22} Arise, O God, plead thine own cause: remember how the foolish man reproacheth thee daily. 
	\item \textbf{Psalm 82:8}  Arise, O God, judge the earth: for thou shalt inherit all nations.
	\item \textbf{Psalm 132:8}Arise, O LORD, into thy rest; thou, and the ark of thy strength.
\end{compactenum}