\subsection{Outlines from Billheimer}

\subsubsection{Who Can Find a Virtuous Woman}

\index[speaker]{Clarence Billheimer!Proverb 20:1, 23:32 (Who Can Find a Virtuous Woman)}
\index[series]{Proverbs (Clarence Billheimer)!Pro 20:1, 23:32 (Who Can Find a Virtuous Woman)}
\index[event]{Mothers' Day (Clarence Billheimer)!Proverb 20:1, 23:32 (Who Can Find a Virtuous Woman)}
\index[date]{2016/02/22!Proverbs 20:1, 23:32 (Who Can Find a Virtuous Woman) (Clarence Billheimer)}

\textbf{Introduction: }The use of the word wine in the Bible has caused a great stir not just among unbelievers, but also those who profess to be saved.  We have a different definition of it than the Bible does, that is the problem.  In the Bible the word translated wine can also mean simple grape juice or jam or jelly.  And those who wish to justify their consumption of alcohol have found what they think are verses to prove their point.  So what is the Bible stand on this controversial issue?  First, careful study, rightly dividing the word of truth, is a must (2Ti 2:15).  From there, we can come to only one conclusion:  abstinence! \\
\\
\begin{compactenum}[I.]
    \item \textbf{Deceivableness of alcohol} 
    \begin{compactenum}[A.]
    	\item It is commercialized in deceit—the ads promote good times, no family problems, etc.
    	\item It is consumed in darkness (bars, restaurants which serve it are always dimly lighted)—compare John 3:18.
    \end{compactenum}
    \item \textbf{Drunkenness and alcohol} 
    \begin{compactenum}[A.]
    	\item First occurrence—Noah’s nakedness, and the shame resulting from it
    	\item Lot’s daughters—getting their father drunk and committing incest with him
    	\item  The priests and Levites were to abstain from alcohol when going to the tabernacle (Lev 10:9).  Application—our body is the tabernacle of the Holy Spirit and we are spiritual priests.
    	\item The Nazarites were to abstain from anything associated with grapes (Num 6:2).
    	\item Drunkenness and nakedness are Siamese twins.
    \end{compactenum}
    \item \textbf{Debates about alcohol} 
    \begin{compactenum}[A.]
    	\item Did Jesus create alcoholic wine at the marriage of Cana?  NO!  “Good wine” = wine not the fermented type.  Same Greek word used by Jesus when He called Himself the Good Shepherd.
    	\item Was Paul commending alcoholic wine to Timothy?  If he was, it was strictly a medical prescription, not a license for social drinking or careless use of it.
    	\item Jesus was accused of being a winebibber—are we to infer from a statement of UNGODLY enemies of Jesus that He was a social drinker?  NO, here is just an example of inspiration when a false statement in and of itself is included in the Bible.  The statement is not true, the fact it was stated is true.
    \end{compactenum}
    \item \textbf{Destructiveness of alcohol} 
    \begin{compactenum}[A.]
    	\item A person’s character is altered.
    	\item A person’s mind is deluded.
    	\item  A person’s body becomes dependent on it.
    	\item A person’s health becomes deteriorated from it.
    	\item The potential destructiveness outweighs any supposed advantages of it.\\
    \end{compactenum}
\end{compactenum}
\textbf{Conclusion: } `We know this is not a friendly issue among Christians, let alone unsaved, because so many see no harm in a social drink now and then.  The problem is, there’s no way to predict it won’t go from there to the destructiveness we saw in this lesson.  Besides, there are so many other good things one can drink besides beer, wine, whiskey, etc.

\index[FACEBOOK]{FUNDAMENTAL BAPTIST SERMON OUTLINES!Clarence Billheimer - Proverb 20:1, 23:32!2016/02/22}



\subsubsection{Train Up a Child}

\index[speaker]{Clarence Billheimer!Proverb 20:11, 22:6 (Train Up a Child)}
\index[series]{Proverbs (Clarence Billheimer)!Pro 20:11, 22:6 (Train Up a Child)}
\index[event]{Mothers' Day (Clarence Billheimer)!Proverb 20:11, 22:6 (Train Up a Child)}
\index[date]{2018/03/07!Proverbs 20:11, 22:6 (Train Up a Child) (Clarence Billheimer)}

\textbf{Introduction: }I heard that Nikita Kruschev said if he could have a child under his training for enough time, he would have him totally convinced that the teachings and operations of Communism were right and have that child for life.  I have heard many preachers point out the fact that children so easily accept things they are told; their faith is so tender and simple.  No wonder Jesus often talked about children.  No wonder He was disturbed when the disciples wanted to turn them away.  It has also been statistically shown that the age ranges when most people trust Christ are the younger ones.
 It is so important for us in this church to train our children.  Here we have a great group of teachers in our church who want you to be here every week so you can be trained how to be the child God wants you to be.  Some of you get support in that training at home maybe better than others.  As I thought about what God would want me to share with you today about training you up as a child, I found fifteen things which I would like each of you to think about.  How well have I learned how to be that way, to do those things?  There are three in each letter of the word child. \\
\\
\begin{compactenum}[I.]
    \item \textbf{C} 
    \begin{compactenum}[A.]
    	\item Christ—the most important person to learn about
    	\item Confession—willing to admit to God and others when you’ve done wrong.
    	\item Compassion—not making fun of others and trying to help in ways you can.
    \end{compactenum}
    \item \textbf{H} 
    \begin{compactenum}[A.]
    	\item Holiness—trying to live the best life you can everywhere and with everyone
    	\item Honor—honor your parents, those in authority over you
    	\item Humility—not bragging so much about how good you are or better than someone else
    \end{compactenum}
    \item \textbf{I} 
    \begin{compactenum}[A.]
    	\item Intercession—yes, you can pray!  You can talk to God just like you talk to family and friends!
    	\item Integrity—keep your promises you make so people will depend on you.
    	\item  Instruction—be willing to learn from others and do your best in school.
    \end{compactenum}
    \item \textbf{L} 
    \begin{compactenum}[A.]
    	\item Love—genuine concern, willing to put others more important than yourself, not always expecting that in return
    	\item Loyalty—be someone that people can depend on to be where or when you’re needed.
    	\item Leadership—if you can be a leader, be one people will want to follow, not feel like they are forced to follow.
    \end{compactenum}
    \item \textbf{D} 
    \begin{compactenum}[A.]
    	\item Discipline—it is very important you realize why you’re disciplined (Prv 22:15; 23:13; 29:15).
    	\item Devotion—be true to those you live with, play with, go to school with in the right ways.
    	\item  Dedication—decide as early as you possibly can what you want to be when you get older.\\
    \end{compactenum}
\end{compactenum}
\textbf{Conclusion: } You know, the Bible tells us when we receive Jesus, we become His child.  And it does not matter how old a person is when that happens, he is a spiritual child in God’s family, and that person needs to realize how important the verses we began with as much as it is for boys and girls to learn that.  We need to thank the Lord for those willing to teach these things to you from the Bible!

\index[FACEBOOK]{FUNDAMENTAL BAPTIST SERMON OUTLINES!Clarence Billheimer - Proverb 20:11, 22:6!2018/03/07}







