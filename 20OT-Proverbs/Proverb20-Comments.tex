\section{Proverb 20 Comments}

\subsection{Numeric Nuggets}
\textbf{13:} There are 13 words in verses 7 and 28. There are 13 unique words in verses 9 and 27. The 13-letter word ``understanding'' is found in the chapter.

\subsection{Proverb 20:9}
No one can honestly say this. We don't know the depths of our hearts, and really shouldn't want to.Jeremiah 17:9, 10 come to mind.\footnote{\textbf{Jeremiah 17:9,10} - The heart is deceitful above all things, and desperately wicked: who can know it? [10] I the LORD search the heart, I try the reins, even to give every man according to his ways, and according to the fruit of his doings.}

\subsection{Proverb 20:12}
What about these eyes and ears? The ears appear seven times in Revelation 2-3, one for each of the churches (Rev 2:7, Rev 2:11, Rev 2:17, Rev 2:29, Rev 3:6, Rev, 3:13, and Rev 3:22). This tells us there are specific messages directed toward those that have the ears to hear. Deuteronomy 29:4 tells us that ``ears to hear'' and ``eyes to see'' are given by the Lord. Ezekiel 12:2 tells us that Israel was granted these, but failed to use them! Matthew 11:15 tells us that the ``ears to hear'' are used to interpret events in the light of scripture. The ``ears to hear'' and ``eyes to see'' are associated with understanding the parables of the kingdom.(Matthew 13:9, 43, Mark 4:9, 23, 7:16, Luke 8:8, 14:35). \footnote{\textbf{Deuteronomy 29:4} - Yet the LORD hath not given you an heart to perceive, and eyes to see, and ears to hear, unto this day.}\footnote{\textbf{Ezekiel 12:2} - Son of man, thou dwellest in the midst of a rebellious house, which have eyes to see, and see not; they have ears to hear, and hear not: for they are a rebellious house.}\footnote{\textbf{Matthew 11:14-15} - And if ye will receive it, this is Elias, which was for to come. [15] He that hath ears to hear, let him hear.}

\subsection{Proverb 20:22}
It is interesting that the word ``recompense'' used in this verse and 26 times in the KJV, is a verb, and the word ``recompence'', used 20 times, is a noun.

\subsection{Proverb 20:25}
Standard interpretations of this verse include (1) a man using something for himself that  as set aside for God and (2) a man making a vow to God and then looking afterwards for a way out of that vow. Here are the doctrines of sanctification and of ``rash vows.'' In this verse, both phrases have to do with things set aside, or ``sanctified''. In the first part, things that are already made holy, or declared holy'' are devoured in a selfish manner. In the second part, things are set apart on a voluntary basis. There is a distinction between ``sacred'' and ``profane.'' Hear is a good rule of thumb: treat things that God has made sacred as sacred, and be careful what you personally add to the list.
There are Old Testament examples and wisdom. Ecclesiastes tells us to take vows to God seriously.\footnote{\textbf{Eclesiates 5:4-6} - When thou vowest a vow unto God, defer not to pay it; for he hath no pleasure in fools: pay that which thou hast vowed. [5] Better is it that thou shouldest not vow, than that thou shouldest vow and not pay. [6] Suffer not thy mouth to cause thy flesh to sin; neither say thou before the angel, that it was an error: wherefore should God be angry at thy voice, and destroy the work of thine hands?} Remember Jephthah back in Judges?\footnote{\textbf{Judges 11:30-31, 34-36} - And Jephthah vowed a vow unto the LORD, and said, If thou shalt without fail deliver the children of Ammon into mine hands, [31] Then it shall be, that whatsoever cometh forth of the doors of my house to meet me, when I return in peace from the children of Ammon, shall surely be the LORD’S, and I will offer it up for a burnt offering. [34] And Jephthah came to Mizpeh unto his house, and, behold, his daughter came out to meet him with timbrels and with dances: and she was his only child; beside her he had neither son nor daughter. [35] And it came to pass, when he saw her, that he rent his clothes, and said, Alas, my daughter! thou hast brought me very low, and thou art one of them that trouble me: for I have opened my mouth unto the LORD, and I cannot go back. [36] And she said unto him, My father, if thou hast opened thy mouth unto the LORD, do to me according to that which hath proceeded out of thy mouth; forasmuch as the LORD hath taken vengeance for thee of thine enemies, even of the children of Ammon.} This is prime example, probably, of a rash vow, one that Jephthah later wished he had never made. Vows like these often masquerade as things called ``convictions'' in Baptist churches, many of which are abandoned shortly after getting them.

Consider what is ``holy'' in the New Testament context. For a believer, our eyes and ears should be set apart, our time, our gifts and abilities, our resources, our bodies, the souls in our lives. For pastors, missionaries, and evangelists, God's people are holy and set apart. You should be careful not to devour them and to take your responsibilities seriously.\footnote{\textbf{Acts 20:28} - Take heed therefore unto yourselves, and to all the flock, over the which the Holy Ghost hath made you overseers, to feed the church of God, which he hath purchased with his own blood.} Your mission is to build Christians and not necessarily to build churches.


Think about the Christian who absorbs the teaching of God's word year after year and does nothing with it to change his life and living and neither does he pass these truths along to others.  In this context, the snare would be the Judgment Seat of Christ. Is this you?

% Starter: balance [23], beauty [29], beginning [21], belly [27, 30], blessed [7, 21], blueness  [30], bread [17, 13], buyer [14]